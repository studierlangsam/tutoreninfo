\documentclass[10pt,ngerman]{scrartcl}
\usepackage{lmodern}
\renewcommand{\sfdefault}{lmss}
\renewcommand{\ttdefault}{lmtt}
\usepackage[T1]{fontenc}
\usepackage[utf8]{inputenc}
\usepackage[a4paper]{geometry}
\geometry{verbose,tmargin=2.8cm,bmargin=4cm,lmargin=2.3cm,rmargin=2.3cm,headheight=0cm,headsep=1cm,columnsep=1cm}
\setcounter{secnumdepth}{-2}
\usepackage{babel}
\usepackage{array}
\usepackage{textcomp}
\usepackage{enumitem}
\usepackage{setspace}
\usepackage{tabularx}
\usepackage{multicol}
\usepackage{graphicx}
\usepackage[dvipsnames,svgnames,x11names,hyperref]{xcolor}
\usepackage[unicode=true,pdfusetitle,
 bookmarks=true,bookmarksnumbered=false,bookmarksopen=false,
 breaklinks=false,pdfborder={0 0 0},pdfborderstyle={},backref=false,colorlinks=true]
 {hyperref}

\makeatletter

%%%%%%%%%%%%%%%%%%%%%%%%%%%%%% LyX specific LaTeX commands.
%% Because html converters don't know tabularnewline
\providecommand{\tabularnewline}{\\}

%%%%%%%%%%%%%%%%%%%%%%%%%%%%%% Textclass specific LaTeX commands.
\newlength{\lyxlabelwidth}      % auxiliary length 

%%%%%%%%%%%%%%%%%%%%%%%%%%%%%% User specified LaTeX commands.
\setlist[itemize,1]
{itemsep=0.2em,
parsep=0pt,
partopsep=0pt,
topsep=0.3em}
\setlist[itemize,2]
{itemsep=0.1em,
parsep=0pt,
partopsep=0pt,
topsep=0.1em}
\usepackage{scrlayer-scrpage}
\pagestyle{scrheadings}
\chead{Tutoreninformation Studier Langsam}

\RedeclareSectionCommand[
    beforeskip=.7\baselineskip,
    afterskip=.1\baselineskip]{subsection}
\addtokomafont{subsection}{\large}
\setlength{\parindent}{0pt}

\RedeclareSectionCommand[
    afterskip=.6\baselineskip]{section}

\AtBeginDocument{
    \def\labelitemi{\(\triangleright\)}
}

\makeatother

\newcommand{\multilinecell}[2]{\begin{tabular}{@{}#1@{}}#2\end{tabular}}

\begin{document}

%%%%%%%%%%%%%%%%%%%%%%%%%%%%%%%%%%%%%%%%%%%%%%%%%%%%%%%%%%%%%%%%%%%%%%%%% START

\title{\vspace{-1em}
    Tutoreninfo Studier Langsam 2022\vspace{-0.8em}
}
\maketitle

\section{Räume und Leute}\label{rooms}

% \renewcommand\tabularxcolumn[1]{m{#1}}
\begin{tabularx}{\columnwidth}[H]{r|XXX}
    & \textbf{Euler} & \textbf{Lovelace} & \textbf{Turing} \\
    \hline
    Leute
    & \textbf{Jonas S.}, \textbf{Katharina}, Julia, Tamira, Sarah, Leo, Jens, Cián, Johanna, Yannick, Paul L., Piotr
    & \textbf{Alina}, \textbf{Linus}, Laura, Max, Lea, Deborah, Jonas L., Jonathan, Florian B., Anh
    & \textbf{Dominik}, \textbf{Nadine}, Linda, Karina, Falko, Luca, Peter, Paul Z., Jan-Arne, Leon, Florian G. \\
    \hline
    Mo (sW, 14-20) & \multicolumn{3}{c}{alle zusammen -1.012, -1.013} \\
    Mi (ganztags) & \multicolumn{1}{c}{-1.009} & \multicolumn{1}{c}{-1.011} & \multicolumn{1}{c}{-1.012} \\
    Do (20-) & \multicolumn{3}{c}{alle zusammen -1.011, -1.012, -1.013, -1.014} \\
\end{tabularx}
\vspace{1em}


\begin{multicols}{2}
\section{Allgemeine Informationen}

\subsection{Telefonnummern}

\begin{tabular}[H]{ll}
    Hauptorga & \href{tel:+49 721 48074683}{+49 721 48074683} \\
    Campus Notruf & \href{tel:+49 721 608 3333}{+49 721 608 3333} \\
\end{tabular}


\subsection{Discord}

Der Hauptkommunikationskanal für die O-Phase soll der Discord sein.\\
Der Einladungslink ist: \href{https://discord.gg/bVsUY5D4mM}{discord.gg/bVsUY5D4mM}
Dieser QR-Code kann dafür den Erstis angeboten werden:

\includegraphics[width=\columnwidth]{2022/qr-code.png}


\subsection{Getränke}\label{drinks}

Linus hat Getränke bestellt und die Bezahlung vorgestreckt.
Zum Abholen im Mathebau einfach in der Fachschaft fragen, eine weitere Zahlung ist nicht nötig.
Insbesondere am Montag könnte es sinnvoll sein, vorher anzurufen und sich anzukündigen, damit das schneller geht.

Wie viel für wann geplant ist, ist \href{https://docs.google.com/spreadsheets/d/1p1hiGPHs2fquxRcn8Yz74vaicJAu-LVF3NOmNHuNc6g/edit?usp=sharing}{dieser Tabelle} zu entnehmen.
Die Bestellung ist auf Provision, sodass wir nicht Benötigtes vollständig erstattet bekommen.
Sollten wir merken, dass es knapp wird, können wir auch noch ``Nachbestellen'' bzw. Reserven der Fachschaft nutzen, die dann natürlich auch gezahlt werden müssen.
Bitte achtet darauf, dass die Pfandflaschen zurückgehen, da wir sonst kein Pfand mehr zurückbekommen.
Weißt bitte auch die Erstis darauf hin.

\subsection{Sonstiges}

Man kann als Info Ana und LA für Mathematiker:innen hören. Allerdings ist
anders als früher LA stärker getrennt. Mindestens didaktisch evtl. auch
inhaltlich.

\section{Wochenplan}

\subsection{Übersicht}

Auf der \href{https://studierlangsam.de/wochenplan}{Website} oder auch auf der \hyperref[LastPage]{letzen Seite}.


\subsection{Montag}
% Dominik


\subsubsection{Einkauf}
Jonas kauft Krepband und Eddings für Namensschilder und bringt sie zum Audimax
mit.
Yannick und Piotr holen Hadikobecher von Luca, Schilder von Peter und kaufen Kekse und Snacks für Donnerstag. Diese können sie in den Fachschaftslagerraum im Mathebau lagern.

\subsubsection{Begrüßungsveranstaltung}

Die Begrüßungsveranstaltung findet um 9:00 Uhr im Audimax statt. Wir
dürfen U-Boote sein (müssen aber nicht!), sollten dann aber in der
Rolle des Erstis bleiben. Falls es nicht genug Platz für alle Erstis
gibt, sollten die U-Boote natürlich wieder gehen (Codewort dafür ist
,,InWis raus``). 

Zwischen 9:30 und 9:45 erscheinen wir als Gruppe vor den Audimax ohne das uns
die Erstis drinnen entdecken. Gegen 10:00 fängt die Gruppenvorstellung an.  Wir
sind als 11. bzw. 3. letzte Gruppe nach \emph{Kitmatheinfo.de} dran. Es wird unser
Film gezeigt und Jonas erzählt was zu uns und unseren Wochenplan.

\subsubsection{Erstisammlung}

Nach der Vorstellung warten wir als Gruppe rechts (aus Publikumssicht) von den
Audimaxhörsaal. Sollte die digitale Einteilung funktionieren, holt Linus unsere
Visitenkarten von der Fachschaft und verteilt diese an die Erstis, um sie in
die Untergruppen (Lovelace, Euler und Turing) einzuteilen. Dominik sammelt die
Erstis im Audimax auf und bringt sie zu uns. 

Die Untergruppen geben dann allen Erstis Namensschildern, begrüßen Sie und
\emph{binden sie in Gespräche ein}. Außerdem holen sie sich Bingoblätter und Bälle von Linda.


\subsubsection{Pizza}

Jede Untergruppe zählt wie viele Menschen vegane, vegetarische oder Pizza mit
Schweinefleisch essen wollen. Diese Zahlen müssen schnellstmöglich an Karina
gemeldet werden, sodass sie berechnen kann wie viele Pizza wir wo bestellen.
Diese Pizzen werden dann von Julia und Jonas zum Mathebau bestellt.


\subsubsection{Zensus und Steuereintreibung}

Kurz bevor die Pizza kommt, sammelt Linda von allen Erstis und Tutierenden Geld ein. 10€ wenn sie Pizza wollen; 5€ sonst.
Die sind für die Pizza, die Getränke beim Kennenlernen, am Mittwoch und beim Spieleabend, die Brötchen am Mittwoch sowie Fixkosten gedacht.
Bei der Gelegenheit sammelt sie die Namen, Untergruppenzugehörigkeit und Emails der Erstis, damit Linus die später in die Mailingliste eintragen kann.

Wenn absehbar ist, dass am Dienstag mehr als 30 Personen pro Kleingruppe frühstücken wollen, passt Linus oder Delegierter die Reservierungen (die auf Florian Götzelmann laufen) an.


\subsubsection{Erstiinformierung}

Folgende Allgemeine Infos sollten die Erstis noch erhalten:
\begin{itemize} 
    \item Möglichst alle sollten Discord beitreten. Z.B. Über den QR-Code in
        diesem Dokument
    \item Wochenplan, Kontakt 
    \item Fahrräder mitbringen! Nicht notwendig aber macht flexibler.
    \item Mittwoch Morgen Teller, Tasse und Besteck mitbringen.  
    \item Cocktailabend Freitag 
    \item Bild hochladen für KIT-Card 
    \item Gesonderte anmeldepflichtige \hyperref[lehramt]{Lehramt-O-Phase}
\end{itemize} 

\subsubsection{Kennenlernen}

Für die Kennlernspiele gehen die Untergruppen in den Schlosspark. Die Länge der
Spiele sollten evtl. der Gruppendynamik angepasst werden.  
\begin{itemize}
    \item \emph{Namen-Ball-Spiel:} Die Erstis und Tutor:innen im  Kreis
        aufstellen lassen.  Ein:e Tutor:in fängt an den Ball zu jemandem zu
        werfen und sagt dabei dessen Namen,  dann ist diese Person dran das
        Gleiche zu tun. Dabei soll jeder den Ball genau einmal bekommen bis er
        am Ende wieder beim Anfang angelangt ist. Anschließend wirft man den
        Ball ein paar Mal in der selben Reihenfolge wie zu Anfang bestimmt und
        sagt immer die Namen. Sind die Erstis sicher genug kommt ein zweiter
        Ball hinzu, der die festgelegte Reihenfolge rückwärts durchläuft.
        Falls das immer noch zu leicht ist kann man weitere Bälle mit neuen
        Reihenfolgen einführen (die Reihenfolge legt man wie beim ersten Mal
        ohne die anderen Bälle fest) und diese können dann ebenfalls umgekehrt
        werden (spätestens beim 4. Ball wird es vermutlich chaotisch). Benötigt
        werden pro Gruppe 4 unterscheidbare Bälle.
    \item \emph{Bingo:} 
        Die Felder auf den Bingoblättern mit Namen von
        Anderen füllen, welche die jeweilige Eigenschaft haben.

    \item \emph{In Ecken stellen:} Die Erstis sollen sich bei jeder der
        folgenden Fragen einer Ecke, also einer Antwort, zuordnen. In jeder Ecke
        sollte ein:e Tutor:in stehen, welche:r sich kurz mit den Erstis
        unterhält.
        \begin{enumerate}
            \item Superkraft (Fliegen, Unsichtbar, Gedankenlesen, Zeit zurückdrehen)
            \item Studienfach (Info, Mathe, Lehramt\footnote{Auf die gesonderten \emph{anmeldepflichtigen} Lehramtveranstaltungen hinweisen}, Techno-/Wirtschaftsmathe)
            \item Wie man zur Uni kommt (Fahrrad, ÖPNV, zu Fuß, Auto)
            \item Vegetarisch, Vegan, Flexitarisch, Fleisch
            \item Wasser (mit Kohlensäure, ohne Kohlensäure, nein)
            \item KIT-Karte (hat man schon, hat man noch nicht, nicht mal Bild hochgeladen\footnote{Bitte Ändern})
            \item Wohnheim, WG, Eltern, Allein, keine Wohnung\footnote{evtl. auf \href{https://www.asta-kit.de/notunterkuenfte}{https://www.asta-kit.de/notunterkuenfte} hinweisen}
            \item Was vor Studium (Allgemeinbildendes Gym., Berufliches/Technisches Gym, Ausbildung, FSJ/BFD/FÖJ/FUJ)
            \item Geschwisteranzahl (0, 1, 2, $\geq 3$)
            \item Welche Farbe hat Mathe (blau, rot, grün, schwarz, gelb, andere Farbe)?
            \item Lieblingsjahreszeit 
            \item OS (Windows, MacOS, Linux, nur Tablet/Handy)
            \item Wie viele Programmiersprachen
            \item Alternativstudiengang (Germanistik, Kunstgeschichte, Jura, Wirtschaftswissenschaften)
            \item der/die/das Nutella
            \item Was macht man um 6 Uhr morgens (schon wach, noch wach, nicht aufweckbar, aufweckbar aber böse)
            \item Lieblingsprokrastination (TikTok/YouTube, Videospiele, Serien/Filme, Lesen, auf dem Bett liegen und an die Decke starren)
        \end{enumerate}

    \item \emph{Sortieren} nach folgenden Kategorien:
        \begin{enumerate}
            \item Anfangsbuchstaben
            \item Größe
            \item Oberteilfarbe
            \item Alter
            \item Schuhgröße
            \item Lieblingszahl
            \item Entfernung zur Uni
            \item Haarlänge
            \item Wie viel Zeit in KA verbracht
            \item Stimmhöhe
            \item (Wie viele Liegestützen)
            \item Herkunftsort (Landkarte)
        \end{enumerate}
\end{itemize}


\subsubsection{Campus-Tour}

Jede Untergruppe macht eine Campustour und sollte das folgende Programm, aber
\emph{in unterschiedlicher Reihenfolge} absolvieren:

\begin{itemize}
    \item Infobau
        \begin{itemize}
            \item Abgabekästen
            \item Info-Fachschaft
            \item Info Bib mit GruppenRäumen
            \item ATIS-Accounts machen
            \item Seminarräume
        \end{itemize}
    \item Hörsaal am Fasanengarten
        \begin{itemize}
            \item LA für Infos (Vorlesung + Übung)
            \item HM (Vorlesung + Übung)
        \end{itemize}
    \item Audimax
        \begin{itemize}
            \item GBI und Programmieren
            \item Für Maschinenbau Anwendung: TM1
        \end{itemize}
    \item Mensa
        \begin{itemize}
            \item KIT-Kartenbezahlung und Autoload
            \item Linien-System vorstellen
            \item Auf Cafeteria hinweisen
            \item Anekdotisch: Aufgrund von Stromausfall gab es im irgendwann mal ein paar mal Essen umsonst
        \end{itemize}
    \item Bibliothek
        \begin{itemize}
            \item LernRäume
        \end{itemize}
    \item AKK
        \begin{itemize}
            \item Kaffee
            \item Günstiges Bier und Glühwein
            \item Bar schichten machen/mithelfen
            \item Partys/Schlonz
            \item O-Phest findet hier statt
            \item Abkürzung raten lassen
        \end{itemize}
    \item Daimler und Benz HS
        \begin{itemize}
            \item Ana und LA für Mathematiker im Daimler
            \item Ana Übung und LA (Mathe) Übung im Benz
            \item D Dach, B Boden
        \end{itemize}
    \item Hertz
        \begin{itemize}
            \item IAM Übung
            \item Uni Kino
        \end{itemize}
    \item Ehrenhof mit Grashof-Hörsaal
        \begin{itemize}
            \item Schwer zu finden: 5min. früher da sein
            \item Hier erste IAM Vorlesung
        \end{itemize}
    \item Mathebau
        \begin{itemize}
            \item Abgabekästen
            \item FSM
            \item Seminarräume und Lernecken
            \item Räume für Studierende sind nach innen (falls ihr nach außen geht für
                ein Tutorium z.b. seid ihr falsch)
            \item Anekdotisch im 3. OG sind die Germanisten
        \end{itemize}
    \item SCC
        \begin{itemize}
            \item Hinweis: Führung am Freitag um 16 Uhr
        \end{itemize}
    \item Gaede-Hörsaal
        \begin{itemize}
            \item IAM (Achtung: 1. Vorlesung im Grashof!)
            \item Lernplätze
        \end{itemize}
    \item Gerthsen-Hörsaal
        \begin{itemize}
            \item Physik Anwendung: Experimentalphysik und Theoretische Physik
        \end{itemize}
    \item Neue Chemie
        \begin{itemize}
            \item Ana 1
            \item Lernplätze
        \end{itemize}
\end{itemize}

\subsubsection{Abendessen}

19 Uhr.

Euler, Lovelace: Charles Oxford
Turing: Veggiezz

Falko geht um 18:00 Uhr zum Treffen der Fachschaft (HS -101).



\subsection{Dienstag}

\subsubsection{Frühstück}

10 Uhr. \\
Euler: intro CAFÉ \\
Lovelace: cafe palaver \\
Turing: Extrablatt

Das ist eine gute Gelegenheit abzumachen wer morgen zum \hyperref[mibreakfast]{Frühstück} was mitbringt (Aufstrich etc).

\subsubsection{Formalitäten}

ca. 13 Uhr. \\
KIT-Karten werden hier abgeholt: \\
Verwaltungsgebäude 10.11, Raum 214 \\
Englerstraße 13
(\href{https://goo.gl/maps/qhAjKJah3wpiFEzq5}{Google Maps})

Das ist eine gute Gelegenheit Erstis noch bei anderen liegen gebliebenen Formalitäten zu unterstützen, falls es welche gibt.

\subsubsection{FBI}

Alle Fachbereichsinformationen beginnen um 14 Uhr in diesen Räumen:

\begin{tabularx}{\columnwidth}[H]{ll}
    Bachelor Mathematik & Neue Chemie   \\
    Bachelor Informatik & Audimax       \\
    Master Mathematik   & Infobau, -102 \\
    Master Informatik   & Infobau, -101 \\
\end{tabularx}

Für Lehramt gibt es gesondert am Mittwoch Programm.

Während des Frühstücks werden für alle FBIs Menschen gefunden, die uns benachrichtigen, wenn sie vorbei sind.
Die Kleingruppen bringen ihre Erstis getrennt zu den FBIs.
Treffpunkt für alle, die direkt zur FBI kommen, ist vor dem Audimax.
Diese Erstis könnten auch noch einmal mit einem Schild oder Banner empfangen werden.

\subsubsection{Institutsvorstellungen}

Nur für Master-Erstis relevant, diese bitte darauf hinweisen. \\
Im Mathebau im Raum 2.067 gibt es um 16 Uhr ein Get-Together mit Doktoranden der Mathe-Institute. \\
Zeitgleich stellen sich in -101 im Infobau einige Professoren der Informatik-Institute vor.

\subsubsection{Stadtführung}\label{stadt}

% TODO - Cián
% Bändchen!

\subsubsection{O-Phest}

Zutritt zum Innenbereich (für 400 Personen) gibt es nut mit O-Phasen-Bändchen.
Das gilt auch für Tutor:innen.
Die Bändchen werden während der \hyperref[stadt]{Stadtführung} abgeholt.
Die können dann zum Beispiel vor der Stadttour ausgeteilt werden.

\subsection{Mittwoch}

Für heute sind \hyperref[drinks]{Getränke} bei der Fachschaft bestellt und können abgeholt werden.

\subsubsection{Gemeinsames Frühstück}\label{mibreakfast}

9 Uhr in den \hyperref[rooms]{Räumen} im Mathebau.

Belag bringen die Erstis mit.

Jemand\texttrademark~ besorgt 2 Brötchen pro Person.
Dominik und Leo bringen jeweils ihr Waffeleisen mit.
Jonas bringt Waffelteig mit.

\subsubsection{O-Rallye}

11 - 17 Uhr mit den \hyperref[rooms]{Räumen} im Mathebau als Basis.

Vor Start: \\
Auf Kneipentour am Abend hinweisen.
Es wird angekündigt, dass Donnerstag Abend ein Spieleabend stattfinden wird und Erstis Spiele mitbringen können.
Auf ATIS-Accounts hinweisen.
Lehramt-Ertis werden erneut auf die gesonderten heutigen \hyperref[lehramt]{Veranstaltungen} und die Anmeldung dafür hingewiesen.

Die Fragebögen für die O-Rallye werden von XXX in der Fachschaft
abgeholt und um 11 Uhr an die Erstis (zur Selbstorganisation) übergeben.
Es sollten immer Tutoren für Rückfragen o.ä. im Seminarraum sein.

\subsubsection{Lehramtinformation}\label{lehramt}

\href{https://www.hoc.kit.edu/zlb/Veranstaltungskalender.php/event/46988?}{Veranstaltungsseite}.
Programm und Anmeldung sind von dort zu erreichen.

11:30 - 16:30 Uhr ist dort Programm, danach Grillen.

Gegebenenfalls werden Erstis von einem Tutor dorthin (Geb. 11.10, Engelbert-Arnold-Hörsaal, \href{https://goo.gl/maps/R9WbmtbrKRxdziYY9}{Google Maps}) geführt.


\subsubsection{Kneipentour}

20 Uhr.

Es werden Bars und Kneipen besucht.
Euler startet am Euro, Lovelace am Marktplatz und Turing am Durlacher Tor.
Zur Orientierung dient \href{https://docs.google.com/spreadsheets/d/1Ea5M858ijKzbtYuySIMmbNUEUUtv-4jGpAmmq59vIvc/edit?usp=sharing}{diese Tabelle}.
Wir starten vermutlich in 3-6 Gruppen und werden je nach Abfall immer weiter zusammenführen.
Das Wechseln und der weitere Verlauf werden engmaschig per Chat koordiniert.
Wir versuchen im Shotz zu enden (hat bis 2 Uhr auf).

Als Alternativprogramm finden Spiele im Z10 statt.



\subsection{Donnerstag}

\subsubsection{Frühstück}

10 Uhr. \\
Euler: Extrablatt \\
Lovelace: intro CAFÉ \\
Turing: cafe palaver

\subsubsection{O-Lympia}

13 - 17 Uhr.

Uns liegen noch nicht genug Infos von der Fachschaft vor.
Wenn hier etwas zu tun ist, klären wir das im Discord.

\subsubsection{Spieleabend}

Wir treffen uns um 20 Uhr in den \hyperref[rooms]{Seminarräumen} zum Spieleabend.
Yannick und Piotr haben Montag bereits Knabberzeug besorgt.
Tutoren und Erstis bringen Spiele mit.



\subsection{Freitag}

\subsubsection{Abschlussveranstaltung}

11:30 - 12:30 Uhr.

Pause für uns.

\subsubsection{Mensa}

12:45 Uhr.

Wir gehen gemeinsam mit unseren Erstis in der Mensa essen.
Einige müssen vermutlich noch ihre Karte kodieren und aufladen.
Daran denken, dazukommende vor der Mensa aufzugabeln.

\subsubsection{Aktivitäten}

Wir bieten einige Aktivitäten an und treffen uns dafür um 14:30 Uhr vor der Mensa.
Folgende Personen kümmern sich um die Durchführung:

\begin{tabular}[H]{ll}
    SCC & Linus \\
    Naturkundemuseum & Laura \\
    ZKM & Jonathan \\
    Bouldern & Jonas, Katha, Piotr, Paul \\
    Minigolf & Anh, Jan-Arne \\
\end{tabular}

Parallel läuft der Aufbau vom Cocktailabend.

\subsubsection{Cocktailabend}

Barbetrieb ab 20 Uhr.

Teilnahme nur nach Anmeldung bis Donnerstag.
Es wird zeitlich passend der \href{https://discord.com/channels/739522765677133894/963505261388107846/1030438323086438440}{Anfahrtsplan} verschickt.

Falko, Flo und Alina sind ab 13 Uhr mit dem Stadtmobil von der Fachschaft unterwegs.
Dabei kaufen sie in der Metro ein und holen und bringen unsere Bestände.
Luca hat die K1-Bar gemietet.
Der Aufbau beginnt um 15 Uhr und wird von Max und Luca organisiert.
Alle Helfer der Barschichten sollen ab 19 Uhr der Bareinführung lauschen.
\href{https://docs.google.com/spreadsheets/d/17ycbRMmSfck2oAsiZ9djPUULAgf4-vCJ6Q4EXgZmB0g/edit?usp=sharing}{Schichtenplan}.



\subsection{Samstag}

\subsubsection{Mädels-Brunch}

10 Uhr im Mathebau Keller.

Alle Mädels gerne anwesend.

\subsubsection{Mathe-Treff}

13 Uhr im Mathebau Keller.

Alle Mathes gerne anwesend.

\subsection{Sonntag}

\subsubsection{Wandern}

Bei ausreichend gutem Wetter bieten Paul L. und Yannick Sonntag eine Wanderung um Bad Herrenalb im Schwarzwald an.
Wir fahren gemeinsam mit der Bahn von Karlsruhe Hauptbahnhof aus.
Die Strecke werden ca 15km sein, gut machbar auch ohne Wandererfahrung.
Es gibt eine Bäckerei in Bad Herrenalb und eine Wirtschaft auf dem Weg, an der wir Pause machen werden. \\
Dauer: ca 10 Uhr bis Nachmittag, genaue Abfahrt wird mit Interessierten direkt abgeklärt \\
Festes schuhwerk wird empfohlen, ist aber nicht zwingend notwendig.

Erzählt euren Erstis gern von dem Angebot, bei Interesse wendet euch an Paul oder Yannick, damit sie einen Überblick über die Teilnehmerzahl haben.

\end{multicols}
\label{LastPage}
\end{document}

% TODO zum rum schicken hängen wir einfach den Wochenplan noch hinten dran
